\documentclass[tikz, border=10pt]{standalone}
\usepackage{tikz}
\usepackage{tikz-3dplot} % For 3D plotting

\begin{document}

% Set the viewing angle
% tdplot_main_coords{theta}{phi}
% theta: rotation around z-axis
% phi: elevation angle from xy-plane
\tdplotsetmaincoords{70}{110} % Adjust angles for a good view

\begin{tikzpicture}[scale=2, tdplot_main_coords]

    % Define cone parameters
    \def\h{3} % Height of the pyramid base plane from apex
    \def\r{1} % Radius of the pyramid base circle

    % Draw the coordinate axes (optional, for context)

    % Apex of the pyramid (contact point)
    \coordinate (O) at (0,0,0);
    \fill (O) circle (0.5pt); % Mark the apex

    % Label for the general force at the apex
    \node[below left, xshift=-2pt, yshift=-2pt] at (O) {$ [ f_x \quad f_y \quad f_z ]^\top $}; % Using pmatrix for vector T notation is hard in TikZ node

    % Draw the 6 edges (force vectors) of the pyramid
    \foreach \angle [count=\i] in {0, 60, 120, 180, 240, 300} {
        % Calculate endpoint coordinates for the edge
        \pgfmathsetmacro{\ex}{{\r*cos(\angle)}}
        \pgfmathsetmacro{\ey}{{\r*sin(\angle)}}
        \coordinate (P\i) at (\ex, \ey, \h);

        % Draw the edge as a vector
        \draw[thick, -latex] (O) -- (P\i);

        % Add label to the first edge (angle=0)
        \ifnum\i=1
            \node[right, xshift=30pt, yshift=90pt, midway, font=\tiny] at (P\i) {$ [ \tau_x \quad \tau_y \quad \tau_z ]^\top $}; % Label one specific wrench
        \fi
    }
    \tdplotdrawarc{(0,0,\h)}{\r}{0}{360}{solid, blue, thick}{}

    % Optionally draw the base of the pyramid (dashed)
    \draw[dashed, thin, gray] (P1) -- (P2) -- (P3) -- (P4) -- (P5) -- (P6) -- cycle;

    % Optionally draw the outline of the original cone (thin, gray)
    % \tdplotdrawarc{(0,0,\h)}{\r}{0}{360}{thin, gray}{}
    % \draw[thin, gray] (O) -- (\r, 0, \h);
    % \draw[thin, gray] (O) -- (-\r*0.5, \r*0.866, \h); % Example tangent lines might need adjustment based on view

\end{tikzpicture}

\end{document}