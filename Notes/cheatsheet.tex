% Options for packages loaded elsewhere
\PassOptionsToPackage{unicode}{hyperref}
\PassOptionsToPackage{hyphens}{url}
%
\documentclass[
  8pt]{extarticle}
\usepackage{amsmath,amssymb}
\usepackage{iftex}
\ifPDFTeX
  \usepackage[T1]{fontenc}
  \usepackage[utf8]{inputenc}
  \usepackage{textcomp} % provide euro and other symbols
\else % if luatex or xetex
  \usepackage{unicode-math} % this also loads fontspec
  \defaultfontfeatures{Scale=MatchLowercase}
  \defaultfontfeatures[\rmfamily]{Ligatures=TeX,Scale=1}
\fi
\usepackage{lmodern}
\ifPDFTeX\else
  % xetex/luatex font selection
\fi
% Use upquote if available, for straight quotes in verbatim environments
\IfFileExists{upquote.sty}{\usepackage{upquote}}{}
\IfFileExists{microtype.sty}{% use microtype if available
  \usepackage[]{microtype}
  \UseMicrotypeSet[protrusion]{basicmath} % disable protrusion for tt fonts
}{}
\makeatletter
\@ifundefined{KOMAClassName}{% if non-KOMA class
  \IfFileExists{parskip.sty}{%
    \usepackage{parskip}
  }{% else
    \setlength{\parindent}{0pt}
    \setlength{\parskip}{6pt plus 2pt minus 1pt}}
}{% if KOMA class
  \KOMAoptions{parskip=half}}
\makeatother
\usepackage{xcolor}
\setlength{\emergencystretch}{3em} % prevent overfull lines
\providecommand{\tightlist}{%
  \setlength{\itemsep}{0pt}\setlength{\parskip}{0pt}}
\setcounter{secnumdepth}{-\maxdimen} % remove section numbering
% --- Preamble for Cheatsheet ---
\usepackage[a4paper, landscape, margin=0.5cm]{geometry} % 设置 A4 横向,边距 1cm
\usepackage{amsmath}          % 数学公式增强
\usepackage{amssymb}          % 数学符号 (如 \mathbb)
\usepackage{graphicx}         % 支持插入图片
\usepackage{multicol}         % 支持多栏排版
\setlength{\columnsep}{1em}    % 设置栏间距
\usepackage{parskip}          % 使用段落间距代替首行缩进,更紧凑
\setlength{\parskip}{0.5em plus 0.1em minus 0.1em}
\setlength{\parindent}{0pt}
\usepackage{ctex}             % XeLaTeX 中文支持


% 可选:如果默认字体不满意,可以取消注释并指定字体
% \setmainfont{Noto Serif CJK SC}
% \setsansfont{Noto Sans CJK SC}
% \setmonofont{Noto Sans Mono CJK SC}

\setcounter{secnumdepth}{2}    % 可选:控制章节编号深度,Cheatsheet 可能不需要太深
% --- End Preamble ---
\ifLuaTeX
  \usepackage{selnolig}  % disable illegal ligatures
\fi
\IfFileExists{bookmark.sty}{\usepackage{bookmark}}{\usepackage{hyperref}}
\IfFileExists{xurl.sty}{\usepackage{xurl}}{} % add URL line breaks if available
\urlstyle{same}
\hypersetup{
  hidelinks,
  pdfcreator={LaTeX via pandoc}}

\author{}
\date{}

\begin{document}

\begin{multicols*}{4}

\hypertarget{policy}{%
\section{Policy}\label{policy}}

\textbf{DexGraspNet}:合成数据(Synthetic Data) + 深度学习

\begin{enumerate}
\def\labelenumi{\arabic{enumi}.}
\tightlist
\item
  场景理解:预测每个点 \textbf{抓取可能性(Graspness)},是否是
  \textbf{物体(Objectness)}
\item
  局部特征:不用全局特征(关联性弱、泛化性差),选择 Graspness
  高的地方附近的点云,提取局部特征(\textbf{几何信息})
\item
  条件抓取生成模块:条件生成处理 \((T, R)\)
  多峰分布,然后采样后直接预测手指形态 \(\theta\)
\end{enumerate}

仅处理包覆式抓取(Power Grasp),没处理指尖抓取(Precision
Grasp);主要使用力封闭抓取;折射(Refraction)/ 镜面反射(Specular
Reflection),导致点云质量差。

\textbf{ASGrasp}:深度修复,合成数据 +
监督学习。域随机化、多模态立体视觉、立体匹配(Stereo Matching) 。

\textbf{可供性(Affordance)}:指一个物体所能支持或提供的交互方式或操作可能性,哪个区域、何种方式进行交互。

\textbf{Where2Act}:大量随机尝试 + 标注。学习从视觉输入预测交互点
\(a_p\)、交互方向 \(R_{z|p}\) 和成功置信度
\(s_{R|p}\)。\textbf{VAT-Mart}:预测一整条操作轨迹。

利用视觉输入进行预测:

\begin{itemize}
\tightlist
\item
  \textbf{物体位姿(Object Pose)}:需要模型、抓取标注。
\item
  \textbf{抓取位姿(Grasp
  Pose)}:直接预测抓取点和姿态,无模型或预定义抓取。
\item
  \textbf{可供性(Affordance)}
\end{itemize}

\textbf{运动规划}(+ 启发式规则,如预抓取
Pre-grasp,到附近安全位置再闭合,避免碰撞)

\begin{enumerate}
\def\labelenumi{\arabic{enumi}.}
\tightlist
\item
  \textbf{操作复杂度有限}:难以处理复杂任务,受启发式规则设计限制。
\item
  \textbf{开环执行(Open-loop)}:规划一次,执行到底,闭眼做事。高频重规划可近似闭环。
\end{enumerate}

\textbf{策略学习}:学习 \(\pi(a_t|s_t)\) 或 \(\pi(a_t|o_t)\),实现
\textbf{闭环控制}。

\textbf{BC}:将 \(D = \{(s_i, a_i^*)\}\) 视为监督学习任务,学习
\(\pi_\theta(s) \approx a^*\)。

\textbf{分布偏移}:策略 \(\pi_\theta\)
的错误会累积,导致访问训练数据中未见过的状态(\(p_\pi(s)\) 与
\(p_{\text{data}}(s)\) 不匹配),策略失效。

\begin{enumerate}
\def\labelenumi{\arabic{enumi}.}
\item
  \textbf{改变 \(p_{\text{data}}(o_t)\)}:\textbf{Dataset
  Aggregation(DAgger)}

  训练 \(\pi_i\) \(\Rightarrow\) 用 \(\pi_i\) \textbf{执行(Rollout)}
  收集新状态 \(\Rightarrow\) 查询专家在此状态下的 \(a^*\)
  \(\Rightarrow\) \(D \leftarrow D \cup \{(s, a^*)\}\) \(\Rightarrow\)
  重新训练 \(\pi_{i+1}\)。但是出错才标注,也会影响准确性。
\item
  \textbf{改变
  \(p_{\pi}(o_t)\)(更好拟合)}:从(传统算法)最优解中获取;从教师策略中学习(有
  \textbf{特权信息 Privileged Knowledge})
\end{enumerate}

\textbf{遥操作数据(Teleoperation)}:贵,也存在泛化问题。

非马尔可夫性:引入历史信息,但可能过拟合,因果混淆(Causal Confusion)。

\textbf{目标条件化(Goal-conditioned)}:\(\pi(a|s, g)\),共享数据和知识。但
\(g\) 也有分布偏移问题。

\textbf{IL
局限性}:依赖专家数据、无法超越专家、不适用于需要精确反馈的高度动态 /
不稳定任务。

\begin{center}\rule{0.5\linewidth}{0.5pt}\end{center}

\textbf{离线学习(Offline
Learning)}:指学习过程无法干预数据的产生过程。我们只能使用一个预先收集好的、固定的数据集进行学习。模仿学习中的
BC 就是典型的离线学习。

\textbf{在线学习(Online
Learning)}:指智能体在学习过程中可以主动与环境交互,实时产生新的数据,并利用这些新数据更新自己的策略。强化学习通常可以在线进行。

\textbf{策略梯度定理}:采样 \(N\) 条轨迹
\(\tau^{(i)} \sim p_\theta(\tau)\),然后计算: \[
\nabla_\theta J(\theta) \approx \frac{1}{N} \sum_{i=1}^{N} \nabla_\theta \log p_\theta(\tau^{(i)}) R(\tau^{(i)})
\]

\textbf{请注意,这个梯度表达式中并没有出现奖励函数 \(R(\tau)\) 关于
\(\theta\) 的梯度 \(\nabla_\theta R(\tau)\)。}
\(\nabla_\theta \log p_\theta(\tau) = \sum_{t=0}^{T-1} \nabla_\theta \log \pi_\theta(a_t | s_t)\)
(马尔科夫性拆开简化)

\textbf{环境模型}:包括状态转移概率 \(p(s_{t+1} | s_t, a_t)\) 和奖励函数
\(r(s_t, a_t)\),真实世界一般都拿不到。

\begin{itemize}
\tightlist
\item
  \textbf{Model-Free}:不需要知道环境的模型
\item
  \textbf{Model-Based}:利用神经网络学习环境的模型
\end{itemize}

\textbf{基础策略梯度算法(REINFORCE)}:按照整条轨迹的总回报
\(R(\tau^{(i)})\) 加权,On-Policy。

初始化 \(\theta\),然后循环

\begin{enumerate}
\def\labelenumi{\arabic{enumi}.}
\tightlist
\item
  用 \(\pi_\theta\) 采样 \(N\) 条轨迹 \(\{\tau^{(i)}\}\),计算
  \(R(\tau^{(i)})\)
\item
  \(\hat{g} = \frac{1}{N} \sum_{i=1}^{N} \left[ \left( \sum_{t=0}^{T-1} \nabla_\theta \log \pi_\theta(a_t^{(i)} | s_t^{(i)}) \right) R(\tau^{(i)}) \right]\)
\item
  更新 \(\theta \leftarrow \theta + \alpha \hat{g}\)
\end{enumerate}

\textbf{On-Policy}:数据来自当前策略。通常效果好,\textbf{样本效率低},每次都得重新采样。

\textbf{Off-Policy}:数据可来自不同策略。\textbf{样本效率高},可能不稳定。

\end{multicols*}

\end{document}
